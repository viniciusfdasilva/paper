% -*- mode: latex; TeX-PDF-mode:t; -*-
\documentclass[10pt,times]{report}
\usepackage[letterpaper,text={6.5in,9.5in},centering,nofoot]{geometry}
% use the showframe option in the above line to show the frame
\usepackage{microtype}
%\usepackage{MinionPro}
\setlength{\tabcolsep}{0pt}
\setlength{\parindent}{0pt}
\setlength{\parsep}{0pt}


% Separation between header and body
\newlength{\partgap}
\setlength{\partgap}{.2in}

% Additional separation between sections
\newlength{\sectiongap}
\setlength{\sectiongap}{.6em}

% Separation between entries
\newlength{\entrygap}
\setlength{\entrygap}{.25em}

\newlength{\sectioncolwidth}
\setlength{\sectioncolwidth}{1in}
\newlength{\colgap}
\setlength{\colgap}{.5em}
\newlength{\stuffwidth}
\setlength{\stuffwidth}{\textwidth}
\addtolength{\stuffwidth}{-\colgap}
\addtolength{\stuffwidth}{-\sectioncolwidth}
\addtolength{\stuffwidth}{-.5em}  % minipage margins?

\pagestyle{empty}


% From TeX by Topic
\def\ifEqString#1#2{\def\testa{#1}\def\testb{#2}%
  \ifx\testa\testb}

\newenvironment{rtable}{
  \begin{minipage}{\textwidth}
  }{
  \end{minipage}
}

\newenvironment{rentry}[3][xxx]{
  \begin{minipage}[t]{\hsize}
    \textbf{#2}\ifEqString{#1}{xxx}\relax\else, \textit{#1}\fi
    \hspace{\stretch{1}} #3 \\
  }{
    \removelastskip
  \end{minipage}
  \\[\entrygap]  % Useful for page squeezing
}

\newcommand{\rline}[2]{
  \begin{minipage}[t]{\hsize}
    #1 \hspace{\stretch{1}} #2
  \end{minipage} \\
}

\newenvironment{rsection}[1]{
  \begin{minipage}[t]{\sectioncolwidth}
    \textsc{#1}
  \end{minipage}
  \hspace{\colgap}
  \begin{minipage}[t]{\stuffwidth}
  }{
    \removelastskip
  \end{minipage}
  \\[\sectiongap]
}

\newenvironment{ritemize}{%
  \begin{list}{$\cdot$}{\topsep 0pt \parskip 0pt \partopsep 0pt
      \itemsep 0pt \parsep 0pt}%
}{\end{list}}

%%%%%%%%%%%%%%%%%%%%%%%%%%%%%%%%%%%%%%%%%%%%%%%%%%%%%%%%%%%%%%%%%%

\begin{document}

% Name
\begin{center}
  \LARGE{\sc{Irene Y. Zhang}}
\end{center}
\vspace{2mm}

% Contact info
\begin{tabular*}{\textwidth}{l@{\extracolsep{\fill}}r}
  \texttt{Irene.Zhang@microsoft.com} & \texttt{https://www.irenezhang.net} \\ 
\end{tabular*}

\vspace{\partgap}

% Table
\begin{rtable}
  \begin{rsection}{Education}
    \begin{rentry}{University of Washington}{Seattle, WA}
      \rline{Ph.D. in Computer Science and Engineering}{Sept 2017}
      Advisors: Hank Levy and Arvind Krishnamurthy\\
      Thesis: \textit{Towards a Flexible, High-Performance Operating System
      for Mobile/Cloud Applications}
    \end{rentry}

    \begin{rentry}{University of Washington}{Seattle, WA}
      \rline{M.S. in Computer Science and Engineering}{December 2013}
      Advisors: Hank Levy, Arvind Krishnamurthy, and Steve Gribble\\
      Thesis: \textit{Simplifying Mobile/Cloud Applications with Sapphire}
    \end{rentry}

    \begin{rentry}{Massachusetts Institute of Technology}{Cambridge,
        MA} \rline{M.Eng. in Electrical Engineering and Computer
        Science}{June 2009} Advisors: M. Frans Kaashoek and Jeremy Stribling\\
      Thesis: \textit{Efficient File Distribution in a Flexible, Wide-area
        File System}
    \end{rentry}

    \begin{rentry}{Massachusetts Institute of Technology}{Cambridge, MA}        
      \rline{S.B. in Computer Science and Engineering}{June 2008}
    \end{rentry}
    \vspace{-0.5em}  
  \end{rsection}

  \begin{rsection}{Interests}
    Operating systems, distributed systems, virtualization, and networking
  \end{rsection}
  
  \begin{rsection}{Recent\\Research\\Highlights}
    \begin{rentry}{Demikernel}{SOSP 2021}
      The Demikernel is a new datapath OS and architecture for
      microsecond-scale datacenter systems and kernel-bypass
      devices. Demikernel accommodates heterogenous kernel-bypass
      devices with a flexible library OS architecture and new
      high-level datapath API with an asynchronous I/O interface and
      zero-copy memory semantics for microsecond I/O
      processing. Demikernel implements this API for RDMA, DPDK and
      SPDK devices with new nanosecond-scale I/O stacks in Rust. Once
      ported to Demikernel, microsecond datacenter systems can run
      across different devices with no code changes.
    \end{rentry}
    
    \begin{rentry}{Persephone}{SOSP 2021}
      Persephone is a kernel-bypass OS scheduler designed to minimize
      tail latency for microsecond-scale applications with wide
      service time distributions. Persephone integrates a new
      scheduling policy, Dynamic Application-aware Reserved Cores
      (DARC), that reserves cores for requests with short processing
      times. Unlike existing kernel-bypass schedulers, DARC is not
      work conserving. DARC profiles application requests and leaves a
      small number of cores idle when no short requests are in the
      queue, so when short requests do arrive, they are not blocked by
      longer-running ones. Counter-intuitively, leaving cores idle
      lets DARC maintain lower tail latencies at higher utilization,
      reducing the overall number of cores needed to serve the same
      workloads and consequently better utilizing the data center
      resources.
    \end{rentry}
    
    \begin{rentry}{Persimmon}{OSDI 2020}
          Persimmon is a persistent memory system that presents a new
          persistent state machine abstraction (PSM) to applications,
          which captures both the granularity of atomic persistent
          operations and the data that should be persisted. Persistent
          state machines nicely capture the RPC-processing behavior of
          modern datacenter storage systems. As a result, Persimmon is
          able to converts existing distributed in-memory storage
          systems into persistent, crash-consistent versions with low
          overhead and minimal code changes.
    \end{rentry}

  \end{rsection}    
\end{rtable}

\begin{rtable}
  
  \begin{rsection}{Conference\\Publications}
    \textbf{I. Zhang}, A. Raybuck, P. Patel, K. Olynyk, J. Nelson,
    O. S. Navarro Leija, A. Martinez, J. Liu, A. Kornfeld Simpson,
    S. Jayakar, P. H.  Penna, M. Demoulin, P. Choudhury, and
    A. Badam. \textit{The Demikernel Datapath OS Architecture for
      Microsecond-scale Datacenter Systems.  } Proc. of the ACM
    Symposium on Operating Systems Principles (SOSP). October
    2021. \\\vspace{-0.5em}

    S. Dharanipragada, S. Joyner, M. Burke, A.  Szekeres, J. Nelson,
    \textbf{I. Zhang}, and D. R. K. Ports. \textit{PRISM: Rethinking
      the RDMA Interface for Distributed Systems.}  Proc. of the
    ACM Symposium on Operating Systems Principles (SOSP). October
    2021. \\\vspace{-0.5em}

    M. Demoulin, J. Fried, I. Pedisich, M. Kogias, B. T. Loo,
    L. T. X. Phan, and \textbf{I. Zhang}. \textit{When Idling is
      Ideal: Optimizing Tail-Latency for Highly-Dispersed Datacenter
      Workloads with Persephone.  } Proc. of the ACM Symposium
    on Operating Systems Principles (SOSP). October
    2021. \\\vspace{-0.5em}

    R. Cheng, W. Scott, E. Masserova, \textbf{I. Zhang}, V. Goyal,
    T. Anderson, A. Krishnamurthy, and B. Parno. \textit{Talek:
      Private Group Messaging with Hidden Access Patterns.} Annual
    Computer Security Applications Conference (ACSAC). December
    2020. \\\vspace{-0.5em}

    W. Zhang, S. Shenker, and \textbf{I. Zhang}.  \textit{Persistent
      State Machines for Recoverable In-memory Storage Systems with
      NVRam.}  Proc. of the USENIX Symposium on Operating
    Systems Design and Implementation (OSDI). November
    2020. \\\vspace{-0.5em}
    
    N. Lebeck, A. Krishnamurthy, H. M. Levy, and
    \textbf{I. Zhang}. \textit{End the Senseless Killing: Improving
      Memory Management for Mobile Operating Systems.}  Proc. of
    the USENIX Annual Technical Conference (ATC). July
    2020. \\\vspace{-0.5em}

    H. Li, M. Hao, S. Novakovic, V. Gogte, S. Govindan,
    D. R. K. Ports, \textbf{I. Zhang}, R. Bianchini, H. S. Gunawi, and
    A. Badam. \textbf{ Efficient and Portable Virtual NVMe Storage on
      ARM SoCs.}  Proc. of the International Conference on
    Architectural Support for Programming Languages and Operating
    Systems (ASPLOS). March 2020.\\\vspace{-0.5em}
    
    J. Goldstein, A. Abdelhamid, M. Barnett, S. Burckhardt,
    B. Chandramouli, D. Gehring, N. Lebeck, C.  Meiklejohn,
    U. F. Minhas, R. Newton, R. Peshawaria, T.  Zaccai, and
    \textbf{I. Zhang}.  \textit{ A.M.B.R.O.S.I.A: Providing Performant
      Virtual Resiliency for Distributed Applications.} Proc. of
    the International Conference on Very Large Data Bases (VLDB).
    December 2019. \\\vspace{-0.5em}

    \textbf{I. Zhang}, N. Lebeck, P. Fonseca, B. Holt, R. Cheng,
    A. Norberg, A. Krishnamurthy, H.  M. Levy. \textit{Automating Data
      Management for Wide-area, Reactive Applications.}  In
    Proc. of the USENIX Symposium on Operating Systems Design
    and Implementation (OSDI). November 2016.\\\vspace{-0.5em}

    B. Holt, J. Bornholt, \textbf{I. Zhang}, D. R. K. Ports, M.
    Oskin, L. Ceze. \textit{Disciplined Inconsistency.}  In
    Proc. of the ACM Symposium on Cloud Computing (SoCC).
    October 2016.\\\vspace{-0.5em}

    \textbf{I. Zhang}, N. K. Sharma, A. Szekeres, D. R. K. Ports,
    A. Krishnamurthy. \textit{Building Consistent Transactions with
      Inconsistent Replication.}.  In Proc. of the ACM Symposium
    on Operating Systems Principles (SOSP).  October
    2015.\\\vspace{-0.5em}

    \textbf{I. Zhang}, A. Szekeres, D. Van Aken, I. Ackerman,
    S. D. Gribble, A. Krishnamurthy, H. M. Levy. \textit{Customizable
      and Extensible Deployment for Mobile/Cloud Applications.}  In
    Proc. of the USENIX Symposium on Operating Systems Design
    and
    Implementation (OSDI).  October 2014.\\
    \vspace{-0.5em}

    S. Peter, J. Li, \textbf{I. Zhang}, D. R. K. Ports, D.  Woos,
    A. Krishnamurthy, T. Anderson, T. Roscoe.  \textit{Arrakis: The
      Operating System is the Control Plane.}  In Proc. of the
    USENIX Symposium on Operating Systems Design and Implementation
    (OSDI).  October 2014. \textbf{Best Paper Award}.\\\vspace{-0.5em}

    \textbf{I. Zhang}, T. Denniston, Y. Baskakov, A. Garthwaite.
    \textit{Optimizing VM Checkpointing for Restore Performance in
      VMware ESXi.}  In Proc. of the USENIX Annual Technical
    Conference (USENIX ATC).  June 2013.
  \end{rsection}  
\end{rtable}

\begin{rtable}
  \begin{rsection}    
    \textbf{I. Zhang}, A. Garthwaite, Y. Baskakov,
    K. C. Barr. \textit{Fast Restore of Checkpointed Memory Using
      Working Set Estimation.}  In Proc. of the ACM Conference
    on Virtual Execution Environments (VEE).  March
    2011.\\\vspace{-0.5em}

    D. R. K. Ports, A. Clements, \textbf{I. Zhang}, S. Madden,
    B. Liskov. \textit{Transactional Consistency and Automatic
      Management in an Application Data Cache.}  In Proc. of the
    USENIX Symposium on Operating Systems Design and Implementation
    (OSDI). October 2010.\\\vspace{-0.5em}

    J. Stribling, Y. Sovran, \textbf{I. Zhang}, X. Pretzer, J. Li,
    M. F. Kaashoek, R. Morris. \textit{Flexible, Wide-Area Storage for
      Distributed Systems with WheelFS.}  In Proc. of the USENIX
    Symposium on Networked Systems Design and
    Implementation (NSDI).  April 2009.
  \end{rsection}
  
  \begin{rsection}{Journal\\Publications}
    \textbf{I. Zhang}, N. K. Sharma, A. Szekeres, A. Krishnamurhty,
    D. R. K. Ports. \textit{Building Consistent Transactions with
      Inconsistent Replication.}.  ACM Transactions on Computer
    Systems (TOCS). December 2018. \\\vspace{-0.5em}

    \textbf{I. Zhang}, F. Adib, P. Bailis. \textit{Research for
      Practice: Distributed Transactions and Networks as Physical
      Sensors}. ACM Queue. October 2016.\\\vspace{-0.5em}

    \textbf{I. Zhang}, N. K. Sharma, A. Szekeres,
    D. R. K. Ports, A. Krishnamurthy. \textit{When Is Operation
      Ordering Required in Replicated Transactional Storage?}.  IEEE
    Data Engineering Bulletin. March 2016.\\\vspace{-0.5em}

    S. Peter, J. Li, \textbf{I. Zhang}, D. R. K. Ports, D. Woos,
    A. Krishnamurthy, T. Anderson, T. Roscoe.  \textit{Arrakis: The
      Operating System is the Control Plane.}  ACM Transactions on
    Computer Systems (TOCS). November 2015.
  \end{rsection}
  
  \begin{rsection}{Workshop\\Publications}
    D. Raghavan, P. Levis, M. Zaharia, and
    \textbf{I. Zhang}. \textit{Breakfast of Champions: Towards
      Zero-Copy Serialization with NIC Scatter-Gather.}  Proc.
    of the Workshop on Hot Topics in Operating Systems (HotOS). June
    2021.\\\vspace{-0.5em}

    A. Ousterhout, A. Belay, and \textbf{I. Zhang}. \textit{Just In
      Time Delivery: Leveraging Operating Systems Knowledge for Better
      Datacenter Congestion Control.  } Proc. of the Workshop on
    Hot Topics in Cloud Computing (HotCloud). Renton, WA. July
    2019. \\\vspace{-0.5em}

    \textbf{I. Zhang}, J. Liu, A. Austin, M.  L. Roberts, and
    A. Badam.  \textit{I'm Not Dead Yet! The Role of the Operating
      System in a Kernel-Bypass Era. } Proc. of the Workshop on
    Hot Topics in Operating Systems (HotOS). Bertinoro, Italy. May
    2019.\\\vspace{-0.5em}

    A. Szekeres and \textbf{I. Zhang}. \textit{Making Consistency More
      Consistent: A Unified Model for Coherence, Consistency and
      Isolation}. In Proc. of the Workshop on Principles and
    Practice of Consistency for Distributed Data (PaPoC). Porto,
    Portugal. April 2018. \\\vspace{-0.5em}

    B. Holt, \textbf{I. Zhang}, D. R. K. Ports, M. Oskin and L. Ceze.
    \textit{Claret: Using Data Types for Highly Concurrent Distributed
      Transactions.} In Proc. of the Workshop on Principles and
    Practice of Consistency for Distributed Data (PaPoC).  Bordeaux,
    France. April 2015.\\\vspace{-0.5em}

    S. Peter, J. Li, D. Woos, \textbf{I. Zhang}, D. R. K. Ports,
    T. Anderson, A. Krishnamurthy, M. Zbikowski. \textit{Towards
      High-Performance Application-Level Storage Management.} In
    Proc. of the USENIX Workshop on Hot Topics in Storage and
    File Systems (HotStorage). Philadelphia, PA. June
    2014.
  \end{rsection}

  \begin{rsection}{Fellowships\\\& Awards}
    \begin{rentry}{UW CSE William Chan Memorial Dissertation}{2018}
      \vspace{-0.75em}
    \end{rentry} 
    \begin{rentry}{Microsoft Research PhD Fellowship}{2015}
      \vspace{-0.75em}
    \end{rentry} 
    \begin{rentry}{Google Anita Borg Memorial Fellowship}{2015}
      \vspace{-0.75em}
    \end{rentry} 
    \begin{rentry}{National Science Foundation Fellowship}{2013}
      \vspace{-0.75em}
    \end{rentry}
    \begin{rentry}{ARCS Foundation Fellowship}{2012}
       \vspace{-0.75em}
    \end{rentry}
    \begin{rentry}{Jeff Dean and Heidi Hopper Endowed Regental Fellowship}{2012}
       \vspace{-0.75em}
    \end{rentry}
    \begin{rentry}{OSDI Best Paper Award}{2014}
      \vspace{-0.75em}
    \end{rentry}
    % \begin{rentry}{CRA Outstanding Undergraduate Award, Honorable
    %   Mention}{2008}
    % \vspace{-0.75em}
    % \end{rentry}
    % \begin{rentry}{Rising Stars Workshop}{2016}
    %   \vspace{-0.75em}
    % \end{rentry} 
    %\begin{rentry}{NCWIT Collegiate Award Runner-up}{2016}
    %  \vspace{-0.75em}
    %\end{rentry} 
    % \begin{rentry}{UW CSE Industrial Affiliates Madrona Prize Runner-Up}{2015}
    %  \vspace{-0.75em}
    % \end{rentry} 
    % \begin{rentry}{Bob Bandes Teaching Award Honorable Mention}{2015}
    %   \vspace{-0.75em}
    % \end{rentry} 
    % \begin{rentry}{UW CSE Industrial Affiliates Madrona Prize}{2014}
    %    \vspace{-0.75em}
    % \end{rentry}
    % \begin{rentry}{National Science Board Annual Meeting Student
    %     Panel}{2013}
    %   \vspace{-0.75em}
    % \end{rentry}
    % \begin{rentry}{VMware Academic Program Top Intern Project}{2008}
    %    \vspace{-0.75em}
    % \end{rentry}
    % \begin{rentry}{Northern Telecom/BNR Award for Best Undergrad. Lab
    %     Project}{2006}
    %   \vspace{-0.75em}
    % \end{rentry}
  \end{rsection}
\end{rtable}

\begin{rtable}

  \begin{rsection}{Service}
    \begin{rentry}{VEE, Program Chair}{2021}
      \vspace{-0.5em}
    \end{rentry}
    \begin{rentry}{OSDI, Program Committee}{2021}
      \vspace{-0.5em}
    \end{rentry}
    \begin{rentry}{SOSP, Program Committee}{2021}
      \vspace{-0.5em}
    \end{rentry}
    \begin{rentry}{EuroDW, Program Chair}{2021}
      \vspace{-0.5em}
    \end{rentry}
    \begin{rentry}{NSDI, Program Committee}{2021}
      \vspace{-0.5em}
    \end{rentry}
    \begin{rentry}{OSDI, Program Committee}{2020}
      \vspace{-0.5em}
    \end{rentry}
    \begin{rentry}{EuroSys, Program Committee}{2020}
       \vspace{-0.5em}
    \end{rentry}
    \begin{rentry}{EuroDW, Program Committee}{2020}
       \vspace{-0.5em}
    \end{rentry}
    \begin{rentry}{APSys, Program Committee}{2020}
       \vspace{-0.5em}
    \end{rentry}
    \begin{rentry}{SOSP, Program Committee}{2019}
       \vspace{-0.5em}
    \end{rentry}
    \begin{rentry}{NSDI, Program Committee}{2019}
       \vspace{-0.5em}
    \end{rentry}
    \begin{rentry}{ASPLOS, Program Committee}{2019}
       \vspace{-0.5em}
    \end{rentry}
    \begin{rentry}{OSDI, Program Committee}{2018}
       \vspace{-0.5em}
    \end{rentry}
    \begin{rentry}{USENIX ATC, Program Committee}{2018}
       \vspace{-0.5em}
    \end{rentry}
    \begin{rentry}{VEE, Program Committee}{2018}
       \vspace{-0.5em}
    \end{rentry}
    \begin{rentry}{ASPLOS, External Reviewer}{2017}
       \vspace{-0.5em}
    \end{rentry}
    \begin{rentry}{OSDI, External Reviewer}{2016}
       \vspace{-0.5em}
    \end{rentry}
    \begin{rentry}{HotCloud, Program Committee}{2016}
       \vspace{-0.5em}
    \end{rentry}
    \begin{rentry}{UW HotPoCSci, Founder}{2015}
       \vspace{-0.5em}
    \end{rentry}
    \begin{rentry}{UW PoCSci, PC Chair}{2015-2016}
       \vspace{-0.5em}
    \end{rentry}
    \begin{rentry}{UW CSE Annual Women's Research Day}{}
      \rline{Committee}{2017}
      \rline{Chair}{2016}
      \rline{Founder}{2015}
       \vspace{-0.5em}
    \end{rentry}
  \end{rsection}
  \begin{rsection}{Teaching}
    \begin{rentry}{Distributed Systems (UW CSE 452)}{Seattle, WA}
      \rline{Teaching Assistant}{Winter 2016}   
      \rline{Teaching Assistant}{Winter 2015}   
      \vspace{-0.5em}   
    \end{rentry}
    \begin{rentry}{Introduction to Operating Systems (UW CSE
        451)}{Seattle, WA}
      \rline{Tutor}{Spring 2016}
      \rline{Tutor}{Fall 2014}
      \rline{Tutor}{Spring 2014}
      \rline{Guest Lecturer}{Fall 2013}
      \rline{Tutor}{Spring 2013}
      \vspace{-0.5em}
    \end{rentry}
    \begin{rentry}{The Hardware/Software Interface (UW CSE
        351)}{Seattle, WA}
      \rline{Tutor, UW Department of CSE}{Winter 2014}
      \rline{Tutor, UW Department of CSE}{Winter 2013}
      \vspace{-0.5em}
    \end{rentry}
    \begin{rentry}{Operating Systems Engineering (MIT
        6.828)}{Cambridge, MA}
      \rline{Teaching Assistant}{Fall 2008}
      % Developed and graded labs assignments where students build an
      % exokernel-style OS. Held weekly office hours to help students
      % with labs and OS fundamentals like virtual memory management,
      % interrupt handlers and process management.
      \vspace{-0.5em}
    \end{rentry}
    \begin{rentry}{Intro. to Digital Systems Lab (MIT
        6.111)}{Cambridge, MA}
      \rline{Teaching Assistant}{Spring 2008}
      % Taught weekly recitations and helped students with labs using
      % FPGAs and Verilog. Helped students design and implement complex
      % final projects such as 3D object tracking. Received student
      % evaluation of 6.3/7.0, one of the highest ratings in the last 5
      % years.
      \vspace{-0.5em}
    \end{rentry}
    \begin{rentry}{Computation Structures (MIT 6.004)}{Cambridge, MA}
      \rline{Lab Assistant}{Spring 2007}
% Held office
%       hours to help students design and build a processor and small OS
%       kernel in simulation.
      \vspace{-0.5em}
    \end{rentry}
    \begin{rentry}{Intro. to Computer Science and
        Programming (MIT 6.00)}{Cambridge, MA}
      \rline{Lab Assistant}{Fall 2006}
 % Taught students
 %      basic computer science concepts such as recursion, abstraction
 %      and OOP.
      \vspace{-1em}
    \end{rentry}
  \end{rsection}


\end{rtable}

\begin{rtable}
  \begin{rsection}{Patents}
    US Patent App. 12/559,484.
    \textit{Saving and Restoring State Information for Virtualized
      Computer Systems.} 
    \textbf{I. Zhang}, K. C. Barr, G. Venkitachalam, I. Ahmad, A. Garthwaite, J. Pool.\\\vspace{-0.5em}

    US Patent App. 13/710,185. 
    \textit{Method for Saving Virtual Machine State from a Checkpoint
    File.}  A. Garthwaite, Y. Baskakov, \textbf{I. Zhang}, K. Christopher,
    J. Pool.\\\vspace{-0.5em}

    US Patent App. 13/710,215. \textit{Method for Restoring Virtual Machine State from a Checkpoint
    File.} A. Garthwaite, Y. Baskakov, \textbf{I. Zhang}, K. Christopher,
    J. Pool.\\\vspace{-0.5em}
    
    US Patent App. 13/935,382. \textit{Identification of Page Sharing Opportunities within Large Pages.} Y. Baskakov, A. Garthwaite, R. Venkatasubramanian, \textbf{I. Zhang}, S. Kim, N. Bhatia, K. Tati\\
  \end{rsection}

  \begin{rsection}{Work\\Experience}
    \begin{rentry}{Microsoft Research}{Redmond, WA}
      \rline{Researcher, Systems Research Group} {Oct 2017 - current}
      \vspace{-.5em}
    \end{rentry}
    \begin{rentry}{VMware, Inc.}{Cambridge, MA}      
      \rline{MTS, Virtual Machine Monitor
        Group}{Jan 2010 - Feb 2013}
      \vspace{-.5em}
    \end{rentry}
    \begin{rentry}{VMware, Inc.}{Cambridge, MA}      
      \rline{R\&D Intern, Virtual Machine Monitor Group}{Jul - Dec 2009}
      \vspace{-.5em}
    \end{rentry}
    \begin{rentry}{VMware, Inc.}{Cambridge, MA}      
      \rline{R\&D Intern, Core Performance Group}{Jun - Aug 2008}
      \vspace{-.5em}
    \end{rentry}
    \begin{rentry}{Quickware Engineering and Design}{Waltham, MA}
      \rline{Engineering Intern}{Jun - Aug 2007} 
      \vspace{-.5em}
    \end{rentry}
    \begin{rentry}{Cummins, Inc.}{Columbus, IN}
      \rline{Engineering Intern, Analysis Led Design}{Jun - Aug 2005}
      \vspace{-.5em}
    \end{rentry}
    \begin{rentry}{Cummins, Inc.}{Beijing, China}
      \rline{International Business Intern}{Jun - Jul 2004} 
      \vspace{-.5em}
    \end{rentry}
    \begin{rentry}{ArvinMeritor, Inc.}{Columbus, IN}
      \rline{Web Development Intern}{Aug 2003 - May 2004}
      \vspace{-.5em}
    \end{rentry}
  \end{rsection}

  % \begin{rsection}{References}
  %   Henry M. Levy\\
  %   Chairman \& Wissner-Slivka Chair\\
  %   Department of Computer Science \& Engineering, University of Washington\\
  %   \texttt{levy@cs.washington.edu}\\
  %   \vspace{-.5em}
    
  %   Arvind Krishnamurthy\\
  %   Professor\\
  %   Department of Computer Science \& Engineering, University of Washington\\
  %   \texttt{arvind@cs.washington.edu}\\
  %   \vspace{-.5em}
    
  %   Thomas E. Anderson\\
  %   Warren Francis \& Wilma Kolm Bradley Chair\\
  %   Department of Computer Science \& Engineering, University of Washington\\
  %   \texttt{tom@cs.washington.edu}\\
  %   \vspace{-.5em}
    
  %   M. Frans Kaashoek\\
  %   Charles Piper Professor\\
  %   Department of Electrical Engineering \& Computer Science, MIT\\
  %   \texttt{kaashoek@csail.mit.edu}\\
  %   \vspace{-.5em}
    
  %   Edward D. Lazowska\\
  %   Bill \& Melinda Gates Chair\\
  %   Department of Computer Science \& Engineering, University of Washington\\
  %   \texttt{lazowska@cs.washington.edu}\\
  % \end{rsection}
\end{rtable}
\end{document}
